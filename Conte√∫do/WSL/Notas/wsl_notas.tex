\documentclass[12pt,a4paper]{article}
\usepackage[brazil]{babel}
\usepackage[utf8]{inputenc}
\usepackage{hyperref}
\usepackage{listings}
\usepackage{xcolor}

\lstset{
    basicstyle=\ttfamily\small,
    keywordstyle=\color{blue},
    commentstyle=\color{gray},
    stringstyle=\color{red},
    showstringspaces=false,
    breaklines=true,
    frame=single
}

\title{Guia de Instalação do Ubuntu no WSL com XFCE4 Terminal}
\author{Produzido com \LaTeX}
\date{\today}

\begin{document}

\maketitle

\section{Introdução}
Este guia descreve como instalar o \textbf{Ubuntu} no \textbf{WSL (Windows Subsystem for Linux)} e configurar o \textbf{XFCE4 Terminal} para um ambiente de terminal mais amigável.

\section{Habilitar o WSL no Windows}

Abra o \textbf{PowerShell} como Administrador e execute:

\begin{lstlisting}[language=bash]
dism.exe /online /enable-feature /featurename:Microsoft-Windows-Subsystem-Linux /all /norestart
dism.exe /online /enable-feature /featurename:VirtualMachinePlatform /all /norestart
wsl --set-default-version 2
\end{lstlisting}

Reinicie o computador se solicitado.

\section{Instalação do Ubuntu}

\subsection{Pela Microsoft Store (Recomendado)}
\begin{enumerate}
    \item Abra a \textbf{Microsoft Store}.
    \item Pesquise por \textbf{Ubuntu}.
    \item Instale a versão desejada (ex.: \textbf{Ubuntu 22.04 LTS}).
\end{enumerate}

\subsection{Via PowerShell (Alternativa sem Store)}
Se preferir, instale via terminal:

\begin{lstlisting}[language=bash]
wsl --install -d Ubuntu
# ou escolha uma versão específica:
# wsl --install -d Ubuntu-22.04
\end{lstlisting}

Após a instalação, abra o Ubuntu no menu iniciar. Na primeira execução será necessário criar um usuário e senha.

\section{Atualização do Sistema}

Dentro do terminal do Ubuntu, execute:

\begin{lstlisting}[language=bash]
sudo apt update && sudo apt upgrade -y
\end{lstlisting}

\section{Instalação do XFCE4 Terminal}

O \textbf{XFCE4 Terminal} oferece uma interface mais amigável que o terminal padrão.

\subsection{Instalar o XFCE4 Terminal}

\begin{lstlisting}[language=bash]
sudo apt install xfce4-terminal -y
\end{lstlisting}

\subsection{Executar o XFCE4 Terminal}

Para iniciar o XFCE4 Terminal no WSL:

\begin{lstlisting}[language=bash]
xfce4-terminal
\end{lstlisting}

\textbf{Observação:} Para usar interfaces gráficas no WSL é necessário um servidor X no Windows, como o \textbf{VcXsrv} ou \textbf{X410}.

\subsection{Instalar e Configurar um Servidor X (Opcional)}

\begin{enumerate}
    \item Instale o \textbf{VcXsrv} no Windows: \url{https://sourceforge.net/projects/vcxsrv/}
    \item Abra o VcXsrv com a configuração \textbf{Multiple Windows}.
    \item No terminal do Ubuntu, exporte a variável de exibição:
\end{enumerate}

\begin{lstlisting}[language=bash]
echo "export DISPLAY=$(cat /etc/resolv.conf | grep nameserver | awk '{print $2}'):0" >> ~/.bashrc
source ~/.bashrc
\end{lstlisting}

Após isso, rode novamente:

\begin{lstlisting}[language=bash]
xfce4-terminal
\end{lstlisting}

\section{Verificação da Versão do Ubuntu}

Para verificar a versão instalada:

\begin{lstlisting}[language=bash]
lsb_release -a
\end{lstlisting}


\end{document}

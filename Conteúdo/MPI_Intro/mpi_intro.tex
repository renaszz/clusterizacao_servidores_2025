\documentclass{beamer}
\usepackage[utf8]{inputenc}
\usepackage[brazil]{babel}
\usetheme{Madrid}

\title{Introdução ao MPI}
\author{Luis Vinicius}
\institute{UniAtenas — Clusterização de Servidores}
\date{Aula 22}

\begin{document}

\begin{frame}
    \titlepage
\end{frame}

\begin{frame}{Objetivo da Aula}
\begin{itemize}
    \item Entender o modelo de troca de mensagens com MPI.
    \item Conhecer os comandos básicos de compilação e execução.
    \item Explorar exemplos simples de programas paralelos.
\end{itemize}
\end{frame}

\begin{frame}{O que é MPI?}
\begin{itemize}
    \item MPI (Message Passing Interface) é um padrão para comunicação entre processos em sistemas distribuídos.
    \item Usado amplamente em computação de alto desempenho (HPC).
    \item Permite comunicação ponto-a-ponto e comunicação coletiva.
\end{itemize}
\end{frame}

\begin{frame}{Modelo MPI}
\begin{itemize}
    \item Cada processo tem memória própria.
    \item A comunicação ocorre por meio de envio e recebimento de mensagens.
    \item Não há memória compartilhada.
\end{itemize}
\end{frame}

\begin{frame}{Comandos Básicos}
\begin{itemize}
    \item Compilar: \texttt{mpicc nome\_arquivo.c -o nome\_executavel}
    \item Executar: \texttt{mpirun -np 4 ./nome\_executavel}
    \item Use o arquivo \texttt{comandos.txt} para revisar os comandos mais usados.
\end{itemize}
\end{frame}

\begin{frame}{Arquivos de Exemplo}
Os seguintes arquivos contêm exemplos ilustrativos:
\begin{itemize}
    \item \texttt{ola\_mundo.c} — exemplo básico de comunicação.
    \item \texttt{envio.c} — envio simples entre processos.
    \item \texttt{vetores.c} — distribuição e soma de vetores.
    \item \texttt{comm\_circular.c} — comunicação em anel.
    \item \texttt{troca\_mensagem.c} — troca entre dois processos.
    \item \texttt{temporizacao.c} — uso de temporização com MPI.
\end{itemize}
\end{frame}

\begin{frame}{Execução em Cluster}
\begin{itemize}
    \item Utilize \texttt{hosts.txt} para indicar os nós do cluster.
    \item Exemplo: \texttt{mpirun --hostfile hosts.txt -np 4 ./ola\_mundo}
    \item Teste local com \texttt{--oversubscribe}, se necessário.
\end{itemize}
\end{frame}

\begin{frame}{Resumo}
\begin{itemize}
    \item MPI é uma poderosa ferramenta de paralelização.
    \item Comunicação explícita entre processos.
    \item Vários exemplos prontos para estudo.
    \item Pratique com os arquivos fornecidos e adapte para seus testes.
\end{itemize}
\end{frame}

\begin{frame}{Próximos Passos}
\begin{itemize}
    \item Modificar os exemplos dados.
    \item Testar em diferentes números de processos.
    \item Medir desempenho com \texttt{temporizacao.c}.
    \item Explorar padrões de comunicação complexos.
\end{itemize}
\end{frame}

\begin{frame}
    \centering
    \Huge Dúvidas?
\end{frame}

\end{document}

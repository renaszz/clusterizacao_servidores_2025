\documentclass{beamer}
\usepackage[brazil]{babel}
\usepackage[utf8]{inputenc}
\usepackage{lmodern}

\title{Comandos Principais do MPI}
\author{Luis Vinicius -- UniAtenas}
\date{Julho de 2025}

\begin{document}

\frame{\titlepage}

\begin{frame}{MPI\_Init}
\textbf{Função:} Inicializa o ambiente MPI. \\
\textbf{Observações:}
\begin{itemize}
    \item Deve ser a primeira chamada em qualquer programa MPI.
    \item Prepara todos os processos para comunicação.
\end{itemize}
\end{frame}

\begin{frame}{MPI\_Finalize}
\textbf{Função:} Finaliza o ambiente MPI. \\
\textbf{Observações:}
\begin{itemize}
    \item Deve ser a última chamada do programa MPI.
    \item Libera recursos usados pelo MPI.
\end{itemize}
\end{frame}

\begin{frame}{MPI\_Comm\_rank}
\textbf{Função:} Retorna o \textit{rank} (identificador) do processo dentro do comunicador. \\
\textbf{Usos:}
\begin{itemize}
    \item Diferenciar a lógica de cada processo.
    \item Controlar o papel de cada processo (ex: mestre vs trabalhadores).
\end{itemize}
\end{frame}

\begin{frame}{MPI\_Comm\_size}
\textbf{Função:} Retorna o número total de processos no comunicador. \\
\textbf{Usos:}
\begin{itemize}
    \item Descobrir quantos processos estão disponíveis.
    \item Controlar loops de comunicação.
\end{itemize}
\end{frame}

\begin{frame}{MPI\_Send e MPI\_Recv}
\textbf{Função:} Envio e recebimento de mensagens ponto-a-ponto. \\
\textbf{Características:}
\begin{itemize}
    \item Comunicação explícita entre dois processos.
    \item Pode causar bloqueios se não coordenado corretamente.
\end{itemize}
\end{frame}

\begin{frame}{MPI\_Bcast}
\textbf{Função:} Envia os mesmos dados de um processo (root) para todos os outros. \\
\textbf{Aplicações:}
\begin{itemize}
    \item Compartilhar dados comuns (ex: parâmetros de entrada).
    \item Reduz complexidade comparado a múltiplos \texttt{Send}.
\end{itemize}
\end{frame}

\begin{frame}{MPI\_Reduce}
\textbf{Função:} Realiza uma operação de redução entre processos (soma, máximo, etc). \\
\textbf{Exemplos de uso:}
\begin{itemize}
    \item Somar os resultados locais em um processo central.
    \item Encontrar mínimo/máximo global.
\end{itemize}
\end{frame}

\begin{frame}{MPI\_Wtime}
\textbf{Função:} Retorna tempo de execução (em segundos). \\
\textbf{Aplicações:}
\begin{itemize}
    \item Medir tempo de comunicação.
    \item Avaliar desempenho de trechos paralelos.
\end{itemize}
\end{frame}

\begin{frame}{MPI\_ANY\_SOURCE e MPI\_ANY\_TAG}
\textbf{Função:} Receber mensagens de qualquer origem ou com qualquer tag. \\
\textbf{Usos:}
\begin{itemize}
    \item Quando a ordem dos remetentes é imprevisível.
    \item Útil em servidores MPI ou balanceamento de carga.
\end{itemize}
\end{frame}

\end{document}

\documentclass{beamer}
\usepackage[utf8]{inputenc}
\usepackage[brazil]{babel}
\usepackage{graphicx}
\usepackage{hyperref}

\title[Clusterização de Servidores]{Clusterização de Servidores: Conceitos e Finalidades}
\author[Prof. Me. Luis Vinicius Costa Silva]{Prof. Me. Luis Vinicius Costa Silva\\Faculdade Atenas}
\date{Aula 1}

\begin{document}

\frame{\titlepage}

\begin{frame}{Plano de Ensino}
\begin{itemize}
    \item Tema: Clusterização de Servidores
    \item Aula 1: Introdução aos Servidores - conceitos e finalidades
    \item Aula 2: Fundamentos da Computação em Cluster
\end{itemize}
\end{frame}

\begin{frame}{O que é um Servidor?}
\begin{itemize}
    \item Equipado com processadores, memória, armazenamento e softwares dedicados.
    \item Pode ser um recurso lógico que processa aplicações e serve dados.
    \item Pode ser físico ou virtual, local ou remoto.
    \item Escalável e com alto poder de processamento.
\end{itemize}
\end{frame}

\begin{frame}{Arquitetura Cliente/Servidor}
\begin{itemize}
    \item Modelo onde clientes acessam dados centralizados em servidores.
    \item Fornece roteamento, controle de acesso e compartilhamento de recursos.
    \item Exemplo: várias estações acessando banco de dados centralizado.
\end{itemize}
\end{frame}

\begin{frame}{Servidores em Redes Locais (LAN)}
\begin{itemize}
    \item Também chamados de servidores dedicados.
    \item Executam aplicações administrativas, banco de dados, backup, etc.
    \item Gabinetes: torre, rack 19", blade.
    \item SOs comuns: Linux (Red Hat, Ubuntu Server), Windows Server.
\end{itemize}
\end{frame}

\begin{frame}{Servidores via Internet}
\begin{itemize}
    \item Instalados em datacenters para prestação de serviços online.
    \item Funções: hospedagem, e-mail, streaming, armazenamento em nuvem.
    \item Requisitos: redundância, balanceamento de carga, alta disponibilidade.
    \item Ex: AWS, Azure, Google Cloud.
\end{itemize}
\end{frame}

\begin{frame}{VM x Container / Docker x Kubernetes}
\textbf{Máquina Virtual (VM)} vs \textbf{Container}:
\begin{itemize}
    \item \textbf{VM}: emula um sistema completo (hardware + SO). Mais pesada.
    \item \textbf{Container}: compartilha o kernel do host. Mais leve e rápida.
\end{itemize}

\vspace{0.5cm}
\textbf{Docker} vs \textbf{Kubernetes}:
\begin{itemize}
    \item \textbf{Docker}: plataforma para criar, empacotar e executar containers.
    \item \textbf{Kubernetes}: orquestrador para gerenciar múltiplos containers em cluster.
\end{itemize}
\end{frame}

\begin{frame}{Tipos Comuns de Servidores}
\begin{itemize}
    \item Aplicação: sistemas acessados por múltiplas estações.
    \item Arquivos: armazenamento e compartilhamento centralizado.
    \item Banco de Dados: alto desempenho para dados transacionais.
    \item Mídia: streaming de áudio e vídeo.
    \item E-mail: envio, recepção e armazenamento.
    \item Backup: cópias seguras de dados.
    \item FTP: upload/download via FTP.
    \item Proxy: filtragem e intermediação de acesso.
    \item Web: hospedagem de sites e aplicações.
\end{itemize}
\end{frame}

\begin{frame}{Softwares e Protocolos Associados}
\begin{itemize}
    \item Web: Apache, NGINX
    \item FTP: FileZilla Server
    \item E-mail: Postfix, Exchange
    \item Backup: Bacula, Veeam
    \item Proxy: Squid
    \item Banco de Dados: MySQL, PostgreSQL, SQL Server
\end{itemize}
\end{frame}

\begin{frame}{Definição de Cluster}
\begin{itemize}
    \item Grupo de computadores interconectados que funcionam como uma única máquina.
    \item Usam redes locais de alta velocidade.
    \item Alternativa de melhor custo-benefício comparado a supercomputadores.
\end{itemize}
\end{frame}

\begin{frame}{Arquitetura de um Cluster}
\begin{itemize}
    \item Nós de computação: executam as tarefas.
    \item Front-end: monitora hardware/software.
    \item Servidor de arquivos: fornece dados para os nós.
    \item Rede de serviço: comunicação entre os nós.
    \item Gateway: acesso externo.
\end{itemize}
\end{frame}

\begin{frame}{Vantagens dos Clusters}
\begin{itemize}
    \item Custo-benefício com hardware padrão.
    \item Escalabilidade e paralelismo.
    \item Tolerância a falhas e manutenção facilitada.
    \item Flexibilidade e suporte a programação paralela.
\end{itemize}
\end{frame}

\begin{frame}{Tipos de Clusters}
\begin{itemize}
    \item HPC (Computação de Alto Desempenho)
    \item HA (Alta Disponibilidade)
    \item Balanceamento de Carga
\end{itemize}
\end{frame}

\begin{frame}{Balanceamento de Carga}
\begin{itemize}
    \item Distribuição de tarefas entre nós para evitar sobrecarga.
    \item Níveis: aplicação, transporte, rede.
    \item Usado em servidores web/FTP de alta demanda.
    \item Algoritmos: Round Robin, Least Connections, etc.
\end{itemize}
\end{frame}

\begin{frame}{Ambiente de Usuário HPC}
\begin{itemize}
    \item SO: Linux, Unix
    \item Acesso: SSH \ (ssh usuario@servidor)
    \item Transferência: SCP, SFTP
    \item Escalonadores: Slurm, PBS
    \item Compiladores: GCC, Intel, PGI
    \item MPI: OpenMPI, MPICH
    \item Linguagens: C, C++, Fortran, Python, R
\end{itemize}
\end{frame}

\begin{frame}{Comandos Unix Básicos}
\begin{itemize}
    \item Arquivos e diretórios: ls, cd, mkdir, rm
    \item Permissões: chmod, chgrp
    \item Visualização: cat, less, head, tail
    \item Editor: vi nome-do-arquivo
\end{itemize}
\end{frame}

\begin{frame}{Transferência de Arquivos}
\begin{itemize}
    \item SCP: \texttt{scp arquivo.txt usuario@host:/caminho}
    \item SFTP: \texttt{sftp usuario@host}
    \item Cliente gráfico: FileZilla
\end{itemize}
\end{frame}

\end{document}

\documentclass{beamer}
\usetheme{Madrid}
\usecolortheme{default}
\usepackage[utf8]{inputenc}
\usepackage[brazil]{babel}

\title[Ubuntu no WSL]{Instalação do Ubuntu no WSL e Configuração do XFCE4 Terminal}
\author{Me. Luis Vinixius}
\date{\today}

\begin{document}

%---------------------------------
\begin{frame}
    \titlepage
\end{frame}

%---------------------------------
\begin{frame}{Agenda}
    \tableofcontents
\end{frame}

\section{Introdução}
\begin{frame}{O que é o WSL?}
\begin{itemize}
    \item \textbf{WSL}: Windows Subsystem for Linux.
    \item Permite rodar distribuições Linux diretamente no Windows.
    \item Ideal para desenvolvimento, ensino e testes rápidos.
\end{itemize}
\end{frame}

\section{Habilitar o WSL}
\begin{frame}{Ativando o WSL no Windows}
Abra o PowerShell como Administrador e execute:

\begin{verbatim}
dism.exe /online /enable-feature /featurename:Microsoft-Windows-Subsystem-Linux /all /norestart
dism.exe /online /enable-feature /featurename:VirtualMachinePlatform /all /norestart
wsl --set-default-version 2
\end{verbatim}

Reinicie o computador se solicitado.
\end{frame}

\section{Instalação do Ubuntu}
\begin{frame}{Instalando o Ubuntu}
\textbf{Opção 1 - Microsoft Store:}
\begin{enumerate}
    \item Abra a Microsoft Store.
    \item Pesquise por "Ubuntu".
    \item Instale a versão desejada (22.04 ou 24.04 LTS).
\end{enumerate}

\textbf{Opção 2 - PowerShell:}
\begin{verbatim}
wsl --install -d Ubuntu
# ou escolha uma versão:
wsl --install -d Ubuntu-22.04
\end{verbatim}
\end{frame}

\section{Primeiros Passos}
\begin{frame}{Atualização do Sistema}
Após abrir o Ubuntu pela primeira vez, execute:

\begin{verbatim}
sudo apt update && sudo apt upgrade -y
\end{verbatim}
\end{frame}

\section{XFCE4 Terminal}
\begin{frame}{Instalação do XFCE4 Terminal}
\textbf{Instalação:}
\begin{verbatim}
sudo apt install xfce4-terminal -y
\end{verbatim}

\textbf{Execução:}
\begin{verbatim}
xfce4-terminal
\end{verbatim}

\textbf{Obs.:} É necessário um servidor X para exibir a interface gráfica.
\end{frame}

\begin{frame}{Configurando o Servidor X}
1. Instale o VcXsrv: \texttt{https://sourceforge.net/projects/vcxsrv/}  
2. Inicie com a opção "Multiple Windows".  
3. Configure o DISPLAY no Ubuntu:

\begin{verbatim}
echo "export DISPLAY=$(cat /etc/resolv.conf | grep nameserver | awk '{print $2}'):0" >> ~/.bashrc
source ~/.bashrc
\end{verbatim}

Depois rode novamente:

\begin{verbatim}
xfce4-terminal
\end{verbatim}
\end{frame}

\section{Verificação}
\begin{frame}{Verificando a Versão do Ubuntu}
\begin{verbatim}
lsb_release -a
\end{verbatim}
\end{frame}

\begin{frame}
\centering
\Huge Obrigado!\\[0.3cm]
\Large Perguntas?
\end{frame}

\end{document}
